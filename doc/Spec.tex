\documentclass[11pt, oneside]{article}   	% use "amsart" instead of "article" for AMSLaTeX format
\usepackage{geometry}                		% See geometry.pdf to learn the layout options. There are lots.
\geometry{letterpaper}                   		% ... or a4paper or a5paper or ... 
%\geometry{landscape}                		% Activate for for rotated page geometry
%\usepackage[parfill]{parskip}    		% Activate to begin paragraphs with an empty line rather than an indent
\usepackage{graphicx}				% Use pdf, png, jpg, or eps� with pdflatex; use eps in DVI mode
								% TeX will automatically convert eps --> pdf in pdflatex		
\usepackage{amssymb}
\usepackage{listings}
\usepackage{enumerate}

\title{Language Specification}
\author{Patrick Vatterott}
%\date{}							% Activate to display a given date or no date

\begin{document}
\maketitle
\section{Basic Types}
\subsection{Identifier}


\subsection{Numeric Literal}
Numeric literals can be expressed in two ways:
\begin{enumerate}[1)]
\item 
\textbf{Decimal:} Decimal literals are expressed as base 10 integers. Formally: 
\begin{lstlisting}
$'0' | '1'..'9' ('0'..'9')*$
\end{lstlisting}
\item 
\textbf{Double:} Double literals are expressed in base 10 and must contain a decimal. Formally: 
\begin{lstlisting}
$('0'..'9')+ '.' ('0'..'9')*$
\end{lstlisting}
\end{enumerate}

\subsection{Identifier}
A variable/function name, or identifier, takes the form:

\begin{lstlisting}
$LETTER (LETTER | '0'..'9')*;$ 

$LETTER$ is defined as: $('\$' | 'A'..'Z' | 'a'..'z' | '_');$
\end{lstlisting}

The placeholder <id> is used to indicate an identifier.

%---------
\section{Functions}
This language supports function definitions and calls that are non-recursive and in the same file.
Function definitions are of the form:

\begin{lstlisting}
<type> <id>(<params>) {

}
\end{lstlisting}

Where <params> is a comma separated list of <type> <id> declarations. For example: 

\begin{lstlisting}
int foo(int a, double b, int c) {

}
\end{lstlisting}

Functions do not need to be declared above where they are used. The file has a comprehensive function namespace.

Recursive and mutually recursive functions are not allowed. The graph of function calls with main as the root node must form a directed acyclic graph (DAG).

%--------------
\section{Expressions}
\subsection{Expressions}

An expression can be an identifier, a literal, or a binary expression.
Binary expressions take the form:
\begin{lstlisting}
  (<id>|<literal>) <op> (<id>|<literal>)
\end{lstlisting}

\begin{center}
\begin{tabular}{|l l l l l l|}
\hline
$<$ & $>$ & $<=$ & $>=$ & $==$ & $!=$ \\
\hline
$+$ & $-$ \\
\hline
$*$ & $/$ \\
\hline
\end{tabular}
\end{center}

\subsection{Assignment Expressions}

We support assignments of the form:
\begin{lstlisting}
  <id> = <expression>;
  <id> += <expression>;
  <id> -= <expression>;
  <id> *= <expression>;
  <id> /= <expression>;
\end{lstlisting}
Expressions of this form will be referred to with the placeholder :
\begin{lstlisting}
  <assignment>
\end{lstlisting}

\section{Control Flow}


\subsection{For}
\begin{lstlisting}
for (<assignment>; <expression>; <assignment>) {

}
\end{lstlisting}

\subsection{While}

\begin{lstlisting}
while (<expression>) {

}
\end{lstlisting}


\subsection{If}
If statements 

\begin{lstlisting}
if (<expression>) {

}
\end{lstlisting}

If statements can also contain an else clause (but no elseif clauses)
\begin{lstlisting}
if (<expression>) {

} else {

}
\end{lstlisting}




\end{document}  