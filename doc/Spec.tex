\documentclass[11pt, oneside]{article}   	% use "amsart" instead of "article" for AMSLaTeX format
\usepackage{geometry}                		% See geometry.pdf to learn the layout options. There are lots.
\geometry{letterpaper}                   		% ... or a4paper or a5paper or ... 
%\geometry{landscape}                		% Activate for for rotated page geometry
%\usepackage[parfill]{parskip}    		% Activate to begin paragraphs with an empty line rather than an indent
\usepackage{graphicx}				% Use pdf, png, jpg, or eps� with pdflatex; use eps in DVI mode
								% TeX will automatically convert eps --> pdf in pdflatex		
\usepackage{amssymb}

\title{Language Specification}
%\author{}
%\date{}							% Activate to display a given date or no date

\begin{document}
\maketitle
\section{Basic Types}
\subsection

\subsection{Numeric Literal}
Numeric literals can be expressed in two ways:
\begin{enumerate}[1)]
\item \textbf{Decimal:} Decimal literals are expressed as base 10 integers. Formally: $'0' | '1'..'9' ('0'..'9')*$
\item \textbf{Double:} Double literals are expressed in base 10 and must contain a decimal. Formally: $('0'..'9')+ '.' ('0'..'9')*$
\end{enumerate}

\subsection{Identifier}
A variable/function name, or identifier, takes the form: $LETTER (LETTER | '0'..'9')*;$ 

where $LETTER$ is defined as: $('\$' | 'A'..'Z' | 'a'..'z' | '_');$

The placeholder <id> is used to indicate an identifier.

%---------
\section{Functions}
This language supports function definitions and calls that are non-recursive and in the same file.
Function definitions are of the form:

<type> <id>(<params>) {

}

Where <params> is a comma separated list of <type> <id> declarations. For example: 

int foo(int a, double b, int c) {

}

Functions do not need to be declared above where they are used. The file has a comprehensive function namespace.

Recursive and mutually recursive functions are not allowed. The graph of function calls with main as the root node must form a directed acyclic graph (DAG).

%--------------
\section{Expressions}
\subsection{Expressions}



\subsection{Conditional Expressions}
\subsection{Assignment Expressions}

We support assignments of the form:
  <id> = <expression>;
  <id> += <expression>;
  <id> -= <expression>;
  <id> *= <expression>;
  <id> /= <expression>;

\section{Control Flow}
\subsection{For}
\subsection{While}
\subsection{If}





\end{document}  